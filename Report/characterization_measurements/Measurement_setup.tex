\pdfbookmark[1]{Measurement setup}{measurement setup}
\chapter{Measurement setup}

\begin{figure}
    \centering
    \includegraphics[width = 13 cm]{Images/measurement_setup.jpg}
    \caption{This is an overview of the measurement setup}
    \label{fig:measurement_setup}
\end{figure}

\chapter{2 probe measurements}
we want to do the 2 probe measurements of the Josephson junctions in order to find the resistance. 

according to Manucharyan2009, we want that the superconducting empedance quantum to be around 1 kOhm \Cite{Manucharyan2009}. Further he states that the 

\begin{equation}
Z_{J} \leq R_{Q}    
\end{equation}


\chapter{Resonators}
We are interrested int he transmission of the resonators
Measurement of resonator peaks are fitted with a lorantzian. \Cite{Manucharyan2009}



\chapter{Measurement of Fluxonium qubit}

According to Manucharyan2009 \Cite{Manucharyan2009} he writes: "The signal is clearly flux-periodic indicating that the junction ring is closed and superconducting. The values of $\Phi$ext at which the signal undergoes full swings correspond to the anticrossings of the 0 - 1 transition frequency of the device with the resonator bare frequency, later inferred to be 8.1755 GHz." I guess that this means that there should be flux periodicity!?

They also "infer" the wavefunctions of the first 3 energies in order to? \Cite{Manucharyan2009}


When Manucharyan2009 measures, they also plot the reflected phase " The grey region shows the reflected phase of a single tone, swept close to the resonator bare frequency exhibiting a 50 MHz vacuum Rabi splitting of the resonator with the fluxonium transition 0  - 1. " They do some datatreatment on page 6 that i can maybe look more into when i get to the point where i want to measure Fluxonium qubits. 
