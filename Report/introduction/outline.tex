\pdfbookmark[1]{Thesis Outline}{outline}
\chapter{Outline}

This small "theoretical" paper is 1 out of 3 which in the end, will aid the theoretical part of my masters thesis. The paper is organized as the following: In the background section a brief introduction to superconductivity and BCS theory is found. This is used to describe cooper pairs and the Josephson tunnel junction. Then we build up a formalism to describe superconducting circuits using analytical mechanics and electrodynamics as well as the quantization of these. Based on this, we derive an expression for the Hamiltonian in order to find eigen energies and eigen values of the Fluxonium circuit. In the simulation section the energy levels of Fluxonium is solved numerically  in the flux basis using python. The simulation is made with sliders which can change the different parameters in order to to find the optimal theoretical values for a Fluxonium qubit. 
\\
The aim of this project is to numerically solve the energy levels of a Fluxonium circuit in order to theoretically find the optimal values of the circuit to use as the 2 defined state used as the Fluxonium qubit. 
\\
Introduce how the paper is organized and that you used python in order to simulate the eigen energies and wavefunctions of the qubit Hamiltonian